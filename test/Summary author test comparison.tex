\documentclass[10pt,a4paper,draft]{report}
            \usepackage{geometry}\geometry{a4paper,            left=22mm,
            right=22mm,
            top=25mm,
            bottom=30mm,
            }
            \usepackage[utf8]{inputenc}
            \usepackage[english]{babel}
            \usepackage{amsmath}
            \usepackage{amsfonts}
            \usepackage{amssymb}
            \usepackage{textcomp}
            \setlength\parindent{0pt}
            \begin{document}

\begin{center}
\section*{Summary -- test (author)}
\end{center}


\subsection*{Definitions}

\textbf{A definition}: The author now jumps in with a bunch of definitions which would really have helped to understand the beginning of the text: Blubbla blabberdi blab blub. Blabla bliblabla (p. 2).\\

\textbf{Another definition}: And another: Blubbla blabberdi blab blub. Blabla bliblabla Blubbadi blab bla. Blubbla blabberdi blab blub. Blabla bliblabla. Blubbadi blab bla (p. 2).\\



\subsection*{Outline}

\textbf{Some statement}: 
Here comes a statement: blubbla blabberdi blab blub. Blabla bliblabla. Blubbadi blab bla. Blubbla blabberdi blab blub. Blabla bliblabla (p. 1).\\


\setlength{\leftskip}{1cm}

\textbf{(O1)} \textbf{One objection}\\
This is an objection to that statement: Blubbla blabberdi blab blub. Blabla bliblabla. Blubbadi blab bla. Blubbla blabberdi blab blub. Blabla bliblabla. Blubbadi blab bla (p. 1).\\

\setlength{\leftskip}{0cm}


\setlength{\leftskip}{1cm}

\textbf{(O2)} \textbf{Another objection}\\
Another objection: Blubbla blabberdi blab blub. Blabla bliblabla. Blubbadi blab bla. Blubbla blabberdi blab blub. Blabla bliblabla. Blubbadi blab bla.Blubbla blabberdi blab blub. Blabla bliblabla. Blubbadi blab bla.Blubbla blabberdi blab blub (p. 1).\\

\setlength{\leftskip}{2cm}
\textbf{(A1)} And an answer to that objection: Blubbla blabberdi blab blub. Blabla bliblabla. Blubbadi blab bla. Blubbla blabberdi blab blub. Blabla bliblabla (p. 1).\\

\setlength{\leftskip}{2cm}
\textbf{(A2)} A second answer to the second objection: Blubbla blabberdi blab blub. Blabla bliblabla. Blubbadi blab bla (p. 1).\\

\setlength{\leftskip}{2cm}
$\rightarrow$ This answer has a certain implication: Blubbla blabberdi blab blub. Blabla bliblabla. Blubbadi blab bla. Blubbla blabberdi blab blub. Blabla bliblabla. Blubbadi blab bla (p. 1).\\

\setlength{\leftskip}{3cm}
\textbf{(O)} \textbf{Decisive objection}\\
But wait, there is a decisive objection to that second answer. It goes: Blubbla blabberdi blab blub Blabla bliblabla. Blubbadi blab bla. Blubbla blabberdi blab blub. Blabla bliblabla. Blubbadi blab bla. Blubbla blabberdi blab blub. Blabla bliblabla [...] Blubbla blabberdi blab blub. Blabla bliblabla. Blubbadi blab bla Blubbla blabberdi blab blub. Blabla bliblabla. Blubbadi blab bla. Blubbla blabberdi blab blub Blabla bliblabla. Blubbadi blab bla (p. 1).\\

\setlength{\leftskip}{0cm}

\textbf{Another statement}: 
So now, how about another statement: Blubbla blabberdi blab blub Blabla bliblabla. Blubbadi blab bla. Blubbla blabberdi blab blub. Blabla bliblabla  (p. 1).\\


\setlength{\leftskip}{0cm}

\textbf{(Q1)} This statement leads to two questions. The first one is: Blubbla blabberdi blab blub Blabla bliblabla. Blubbadi blab bla. Blubbla blabberdi blab blub. Blabla bliblabla. Blubbadi blab bla. Blubbla blabberdi blab blub. Blabla bliblabla. Blubbadi blab bla. Blubb  (p. 1).\\

\setlength{\leftskip}{0cm}


\setlength{\leftskip}{0cm}

\textbf{(Q2)} The second one is also rather interesting; for could’nt we ask if blubbla blabberdi blab blub. Blabla bliblabla. Blubbadi blab bla?  (p. 1).\\

\setlength{\leftskip}{0cm}

\textbf{A list to lighten the mood}

\begin{itemize}
\item \textbf{First item}: First item: Blubbla blabberdi blab blub. Blabla bliblabla. Blubbadi blab bla. Blubbla blabberdi blab blub (p. 1).
\item Second item: Blubbla blabberdi blab blub. Blabla bliblabla. Blubbadi blab bla Blubbla blabberdi blab blub. Blabla bliblabla. Blubbadi blab bla. Blubbla blabberdi blab blub [...] The second item continues here: Blubbla blabberdi blab blub Blabla bliblabla. Blubbadi blab bla. Blubbla blabberdi blab blub. Blabla bliblabla. Blubbadi blab bla (p. 1).
\item Third item: Blubbla blabberdi blab blub. Blabla bliblabla. Blubbadi blab bla. Blubbla blabberdi blab blub. Blabla bliblabla. Blubbadi blab bla. Blubbla blabberdi blab blub. Blabla bliblabla Blubbadi blab bla. Blubbla blabberdi blab blub. Blabla bliblabla. Blubbadi blab bla. Blubbla blabberdi blab blub. Blabla bliblabla. Oh no, there is a page break coming up. What are we to do? No biggie. Blubbadi blab bla  (p. 1-2).
\item \textbf{Last item}: Now for the fourth and last item of the list: Blubbla blabberdi blab blub. Blabla bliblabla. Blubbadi blab bla. Blubbla blabberdi blab blub. Blabla bliblabla Blubbadi blab bla. Blubbla blabberdi blab blub. Blabla bliblabla (p. 2).
\end{itemize}

\textbf{Conclusion}: 
And now, here’s actually the conclusion. Clearly, we have seen that blubbla blabberdi blab blub. Blabla bliblabla. Blubbadi blab bla. Blubbla blabberdi blab blub Blabla bliblabla. Blubbadi blab bla. Blubbla blabberdi blab blub. Blabla bliblabla. Blubbadi blab bla. Blubbla blabberdi blab blub. Blabla bliblabla. Blubbadi blab bla. Blubbla blabberdi blab blub. Blabla bliblabla. Blubbadi blab bla. Blubbla blabberdi blab blub. Blabla bliblabla Blubbadi blab bla. Blubbla blabberdi blab blub. Blabla bliblabla. Blubbadi blab bla.1  (p. 2).\\



\end{document}